\section{Inngangur}
-Við gerum hendi sem á að hreyfa sig nákvamlega og örugglega og verður mannleg í útliti hún verður stjórnuð og hreyfð  með  loftþrýstingi, forritunarmálið sem við notum verður C# með Arduino.
-fyrir utan að vera ógeðslega cool þá mun þetta verkefni hafa stór og góð áhrif á mig og samstarfs mann minn sem auglysingargildi og sönnun á hæfni þá verður
-þetta líka notað sem gerfi útlimur
+Við gerum hendi sem á að hreyfa sig nákvæmlega og örugglega og verður mannleg í útliti. Henni verður stjórnað og hreyfð  með  loftþrýstingi. Forritunarmálið sem við notum verður C# með Arduino.
+Fyrir utan að vera ógeðslega cool þá mun þetta verkefni hafa stór og góð áhrif á mig og samstarfsmann minn, sem auglýsingargildi og sönnun á hæfni. Þetta var upprunalega planið en vegna ófyrirséðrar seinkunar á sendinguni frá Danmörku þá þurftum við að skipta um verkefni og ákváðum að gera flokkunarvél sem virkar út frá litum. Við erum ennþá með C# sem forritunartungumálið, þetta hefur mikið notagildi, góð æfing í hönnun og samvinnu. Þetta gæti líka verið notað hjá fyrirtækjum til að flokka næstum hvað sem er út frá litum.
\begin{figure}[h]
\includegraphics[scale=.3]{img/system}
\end{figure}